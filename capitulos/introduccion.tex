\chapter{Introducción}
Grupo Impresor Criterio S.A. de C.V. es una empresa de con sede en la ciudad de Pachuca de Soto, Hidalgo. Su giro principal es el servicio de comunicación y publicación de información político-económico, así como de interés general por medio de periódico, revistas y redes sociales. Dentro de la misma existen distintas áreas como fotografía, deportes, publicidad, bodega, sistemas, contabilidad, recursos humanos, web, contabilidad.

Actualmente la empresa trabaja con SAI ERP, llevando a cabo las funciones de facturación, compras, gastos, cuentas por pagar, proveedores, nómina y almacén. Este sistema se ha estado utilizando desde 4 años. Éste software necesita como minimo 4GB de memoria RAM, un procesador core i3 o superior y 1GB de espacio en  memoria para funcionar correctamente.  Durante este periodo ha presentado ciertas inconsistencias en el área de almacén, ya que para subir al sistema se tienen que llenar formatos a mano para luego ser subidas al sistema. Este proceso desperdicia mucho tiempo y puede ser optimizado.

\section{Antecedentes}
Actualmente existen una gran cantidad de ERP's que son capaces de adaptarse a las necesidades de cualquier organización, además de mejorar el desempeño de las actividades que se realizan dentro de una organización.

\section{Problemática}
Grupo Impresor Criterio S.A. de C.V. es una empresa con el giro de servicios, la cual hace uso de SAI ERP como sistema principal, concentrando las principales actividades de facturación, compras, gastos, cuentas por pagar, proveedores, nómina y almacén, además de la generación de reportes. En el área de almacén se encargan principalmente de recibir todo el material que la empresa utilizara, además de cambiarlo de almacén según requieran el material. Existen ciertos inconvenientes cuando se trata de subir los datos al sistema, ya que cuando llegan revistas o periódicos, se tiene que llenar un formato a mano, para posteriormente subirlo al sistema. Esta actividad puede llevar menos tiempo del que normalmente necesita.

El alcance del presente trabajo es analizar SAI ERP y desarrollar una solución viable, mediante una actualización del sistema o desarrollando un sistema nuevo. Se trabajará bajo la metodología XP (eXtreme Programming) en los lenguajes de C\# para su versión de escritorio y Java para su versión móvil y para el web service, y se usarán las normas IEEE 830 para los requisitos y UML para los diagramas de casos de uso, diagramas de componentes y diagramas de clases.

\section{Objetivo general}
Desarrollar un sistema de control de almacén en los lenguajes de C\# para escritorio y Java para la versión móvil, asi como el Web Service, con el fin de mejorar y agilizar las distintas actividades que se realizan en dicho almacén, automatizar la generación de reportes, agilizar la cobranza y mantener la integridad de los datos almacenados.

\section{Objetivos específicos}
\begin{itemize}
\item Conocer distintos ERP similares.
\item Definir los requerimientos de software.
\item Diseñar la base de datos.
\item Diseñar la interfaz del sistema.
\end{itemize}

\section{Justificación}
SAI ERP es un sistema robusto que cuenta con una gran cantidad de herramientas, y con una interfaz que puede resultar difícil de comprender para el usuario, además de que se puede reducir el tiempo en el que se realizan ciertas tareas como cobranza, captura de datos a través de un dispositivo móvil.