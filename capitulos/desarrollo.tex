\chapter{Desarrollo de Actividades}

\section{Metodología propuesta}
La metodología propuesta para el desarrollo de este proyecto es Programación Extrema (XP).

\subsection{Justificación}
El uso de ésta metodología ágil ofrece múltiples ventajas en el desarrollo de proyectos de software en cuanto a la planeación y pruebas, ya que la tasa de errores es baja, facilita los cambios, fomenta una amplia comunicación entre los desarrolladores y el cliente. Además de que ésta metodología puede ser usada para cualquier lenguaje de programación.

\section{Etapas del proyecto}

\section{Historias de usuario}

\section{Requerimientos}
\subsection{Requerimientos funcionales}
\begin{itemize}
\item Establecer conexión simultanea desde el sistema de escritorio y android a la misma base de datos.
\item Actualización automática de la información a la base de datos.
\item Poner la información en espera si no se ha detectado alguna conexión a internet.
\item Generación automática de reportes.
\end{itemize}

\subsection{Requerimientos no funcionales}
\begin{itemize}
\item Interfaz simple y comprensible para el usuario.
\item Confirmaciones antes y después de actualizar la base de datos.
\item Base de datos de respaldo.

\end{itemize}

\section{Recursos}
Se describen a continuación los recursos que se utilizaron para el desarrollo de este trabajo:

\subsection{Recursos tecnológicos}
\begin{itemize}
\item Visual Studio 2017
\item Android Studio
\item Enterprise Architect
\item Balsamiq Mockups
\item NetBeans
\item MySQL
\end{itemize}

\subsection{Recursos humanos}

\section{Diseño}
El diseño de la interfaz de usuario para la versión de escritorio y android se realizó en Balsamiq Mockups.

\subsection{Diseño del sistema de escritorio}

\subsection{Diseño del sistema de android}

\section{Codificación}

\section{Pruebas}
